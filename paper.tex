\documentclass{article}


\begin{document}

\begin{abstract}

  Herein we share a data set of typical gait data for 11 subjects walking at a
  three speeds on an instrumented treadmill while being longitudinally
  perturbed in the stance phase with psuedo-random fluctuations in the
  treadmill belt speed. We provide raw marker and ground reaction loads in
  addition to processed data that includes gait landmarks, 2D joint angles,
  angular rates, and joint torques. The protocol is described in detail along
  with the the additional meta data about each of the data files. This data can
  be useful for validating or genreating mathematical models that capable of
  simulating non-periodic and pertubed gaits.

\end{abstract}

\section{Introduction}

Emergig powered prothetics are in need of bio-inspired control systems.
Ideally, a powered prosthetic would behave in such a way that the user would
feel like their limb was never disabled. There are a variety of approaches to
developing bio-inspired control systems some of which aim to mimic the
reactions and motion of an abled bodied person. A modern gait lab is able to
collect a variety of kinmetic, kinetic, and physiological data from humans
during gait. This data can be used to drive the design of the controller. With
a rich enough data set, one may be able to identify control mechanism used
during a human's natural gait and recovery from perturbations. We have
collected one such data set that we believe is richer than previous gait data
sets and may be rich enough for control identification. The data can also be
used for verifciation purposes for controllers that have been designed in other
manners.

We collected over five and half hours of perturbed gait data from 11 subjects
sampled at 100 hz. The data has been organized and made available for other
research uses.

\section{Raw Data}

The raw data consists of a set of ASCII tab delimited text files output from
both the Mocap and Record modules in D-Flow in addition a YAML file that
contains all of the necessary meta data for the given trial. These three files
are stored in a hierarchy of directories with one trial per directory. The
directories are named in the following fashion \verb+T001/+ where \verb+T+
stands for ``trial'' and the following three digits are provide a unique trial
identification number.

\subsection{mocap-xxx.txt}

The output from the D-Flow mocap module is stored in a tab delimted file named
\verb+mocap-xxx.txt+ where \verb+xxx+ represents the trial id number. The file
contains several columns:

\begin{description}
  \item[TimeStamp] The system monotonically time when D-Flow recieves a frame
    from Cortex. These are approximately at 100 hz and given in seconds.
  \item[FrameNumber] Monotonically increasing integers that correspond to each
    frame received from Cortex.
  \item[Marker Coordinates] Any column that ends in \verb+.PosX+, \verb+.PosY+,
    or \verb+.PosZ+ are marker coordinates expressed in Cortex's Cartesian
    reference frame. These values are in meters.
  \item[Ground Reaction Loads] The left \verb+FP1+ and right \verb+FP2+ force
    plates output the forces and moments of the BLANK 

Markers

We make use of the full body 47 marker set described in \cite{Ton's HBM paper}.


\end{document}
